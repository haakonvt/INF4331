\documentclass[a4paper]{article}

% Import some useful packages
\usepackage[margin=0.5in]{geometry} % narrow margins
\usepackage[utf8]{inputenc}
\usepackage[english]{babel}
\usepackage{hyperref}
\usepackage{minted}
\usepackage{amsmath}
\usepackage{xcolor}
\definecolor{LightGray}{gray}{0.95}

\title{Peer-review of assignment 4 for \textit{INF3331-HaakonVikor}}
\author{Yrjan Skrimstad, Git-repo INF3331-Yrjan, {yrjansk@ulrik.uio.no} \\
        Arild Lillegård, Git-repo INF3331-Arild, {arildlil@ulrik.uio.no} \\
        Daniel Folgerø, Git-repo INF3331-Daniel, {danielfolgero@gmail.com}}
\date{Deadline: Tuesday, 13. October 2015, 23:59:59.}
\date{Deadline: Tuesday, 13. October 2015, 23:59:59.}

\begin{document}
\maketitle

\section{Review}\label{sec:review}
Test system: Python 2.7.10 - Debian Linux

%%%%%%%%%%%%%%%%%%%%%%%%%%%%%%%%%%%%%%%%%%%%%%%%%%%%%%%%%%
\subsection*{Assignment 4.1: Retrieve web page}
\begin{itemize}
\item You print the error message to stdout with print, perhaps it would be better if you threw an exception or printed the error to stderr with sys.stderr.write().
\item Otherwise it seems to work as intended.
\end{itemize}

%%%%%%%%%%%%%%%%%%%%%%%%%%%%%%%%%%%%%%%%%%%%%%%%%%%%%%%%%%
\subsection*{Assignment 4.2: Find link to location}
\begin{minted}[bgcolor=LightGray, fontsize=\footnotesize]{python}
regular_expr = 'http://www\.yr\.no/place/Norway/[^/]*/[^/]*/' + location + '/forecast.xml'
\end{minted}
The string here could be changed to include an r in front, e.g. r"content". That way you can avoid having to escape certain symbols like dot.\vspace{4mm}


\begin{minted}[bgcolor=LightGray, fontsize=\footnotesize]{python}
if location == '':
\end{minted}
This is not perfect python:\\
\begin{minted}[bgcolor=LightGray, fontsize=\footnotesize]{python}
if not location:
\end{minted}
This would perhaps be a bit nicer:\vspace{4mm}

\begin{minted}[bgcolor=LightGray, fontsize=\footnotesize]{python}
if len(list_of_results) == 0:
\end{minted}
Same here\\
\begin{minted}[bgcolor=LightGray, fontsize=\footnotesize]{python}
if not list_of_results:
\end{minted}
This would be prettier code that does the same.\\\vspace{5mm}

Otherwise all the code seem to work fine and is well documented.

%%%%%%%%%%%%%%%%%%%%%%%%%%%%%%%%%%%%%%%%%%%%%%%%%%%%%%%%%%
\subsection*{Assignment 4.3: Retrieve weather information}
\begin{itemize}
\item retrieve\_weather\_raw\_data() returns raw xml data not parsed data. It seems the assignment text wants parsed data.
\item Code is hard to follow.
\item Documentation is good.
\item We really like the download status functionality!
\end{itemize}

%%%%%%%%%%%%%%%%%%%%%%%%%%%%%%%%%%%%%%%%%%%%%%%%%%%%%%%%%%
\subsection*{Assignment 4.4: Buffer all internet activity}
\begin{itemize}
\item Buffering works nicely, but we're having a hard time testing for obsolete buffering.
\item Buffer does not seem to include the required timestamp argument and instead include a buffer\_is\_valid parameter.
\item Code is nice, well documented and easy to follow.
\end{itemize}

%%%%%%%%%%%%%%%%%%%%%%%%%%%%%%%%%%%%%%%%%%%%%%%%%%%%%%%%%%
\subsection*{Assignment 4.5: Create weather forecast}
\begin{itemize}
\item weather\_update() seems to be working nicely.
\item weather\_update() is large, complex and somewhat hard to follow. Could perhaps be split into several functions ..?
\item Documentation is fine.
\end{itemize}

%%%%%%%%%%%%%%%%%%%%%%%%%%%%%%%%%%%%%%%%%%%%%%%%%%%%%%%%%%
\subsection*{Assignment 4.6: Testing the code}
\begin{itemize}
\item All tests pass.
\item We're uncertain if the doctests in test\_4\_4() and test\_4\_4\_2() are used, and therefore uncertain about the correctness of those tests. In those two tests you could have used py.test's capsys-functionality shown here:\\ \href{https://pytest.org/latest/capture.html}{https://pytest.org/latest/capture.html}.
\end{itemize}


%%%%%%%%%%%%%%%%%%%%%%%%%%%%%%%%%%%%%%%%%%%%%%%%%%%%%%%%%%
\subsection*{Assignment 4.7: Extreme places in Norway}
\begin{itemize}
	\item Code is fine, documentation is fine.
    \item Seems to work as intended.
\end{itemize}

\section{Estimated points}\label{sec:points}
Based on the point system (\href{http://www.uio.no/studier/emner/matnat/ifi/INF3331/h15/assignments/review_rules.pdf}{download link}),
how many points would you give this solution.

For example:
\begin{itemize}
\item -2 points for one to three tests not working/missing. At least we believe two of them not to be working correctly.
\end{itemize}

Total \noindent\textbf{28 out of 30 points}


\bibliographystyle{plain}
\bibliography{literature}

\end{document}
