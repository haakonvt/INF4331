\documentclass[a4paper]{article}

% Import some useful packages
\usepackage[margin=0.5in]{geometry} % narrow margins
\usepackage[utf8]{inputenc}
\usepackage[english]{babel}
\usepackage{hyperref}
\usepackage{minted}
\usepackage{amsmath}
\usepackage{xcolor}
\definecolor{LightGray}{gray}{0.95}

\title{Peer-review of assignment 5 for \textit{INF3331-HaakonVikor}}
\author{Yrjan Skrimstad, Git-repo INF3331-Yrjan, {yrjansk@ulrik.uio.no} \\
        Arild Lillegård, Git-repo INF3331-Arild, {arildlil@ulrik.uio.no} \\
        Daniel Folgerø, Git-repo INF3331-Daniel, {danielfolgero@gmail.com}}
\date{Deadline: Tuesday, 10. November 2015, 23:59:59.}

\begin{document}
\maketitle
\section{Review}\label{sec:review}

Tested on Debian Linux, Python 2.7.10

%%%%%%%%%%%%%%%%%%%%%%%%%%%%%%%%%%%%%%%%%%%%%%%%%%%%%%%%%%
\subsection*{Assignment 5.1:  Python implementation of the heat equation}
The code seems to work nicely.\\
Initiating lists with a value could potentially be done nicer with a list comprehension like this (this is for 0):

\begin{minted}[bgcolor=LightGray, fontsize=\footnotesize]{python}
u = [([0] * n) for y in xrange(m)]
\end{minted}
\\

%%%%%%%%%%%%%%%%%%%%%%%%%%%%%%%%%%%%%%%%%%%%%%%%%%%%%%%%%%
\subsection*{Assignment 5.2: NumPy and C implementations} \label{sec:assignment5.2}
\textbf{Numpy version:}
Seems to perform correctly and efficiently with good use of vectorisation. The documentation could have included information about return values.
\\
\noindent\textbf{Weave version:}
You seem to be assigning more or less the same code string to the same code-variable in three places. This could be done better in a single declaration. The code is quite efficient, though it might be more efficient if you find a way to skip copying the array on every time iteration. Perhaps with a pointer swap?
\\
\noindent\textbf{Instant version:}
This is almost identical to the weave-version, but without the plot. There's not much to be said.
\\
\noindent\textbf{In general:}
You could probably have moved the plotting function out of the solver-functions. Currently you have basically the same plotting code several places, instead of having it one callable place. This would be easier to read and easier to maintain.

%%%%%%%%%%%%%%%%%%%%%%%%%%%%%%%%%%%%%%%%%%%%%%%%%%%%%%%%%%
\subsection*{Assignment 5.3: Testing}
Testing seems to work correctly and is implemented with py.test. However this could be done even nicer if you named the test file `test\_something.py`. That will let py.test automatically find the file so you don't have to write the filename when calling py.test.


%%%%%%%%%%%%%%%%%%%%%%%%%%%%%%%%%%%%%%%%%%%%%%%%%%%%%%%%%%
\subsection*{Assignment 5.4:  Develop a user interface}
On this system the plot disappears the moment it appears. This could probably be solved with a raw\_input() call like in the other files.
Other than that it seems the UI works as intended.\\
\indent However the file is somewhat hard to read. Especially with some lines that are quite a bit too long. You should be able to split up long lines in a string by using something like this.

\begin{minted}[bgcolor=LightGray, fontsize=\footnotesize]{python}
really_long_line = "this line is really really really really really really" \
                   " long"
\end{minted}


%%%%%%%%%%%%%%%%%%%%%%%%%%%%%%%%%%%%%%%%%%%%%%%%%%%%%%%%%%
\subsection*{Assignment 5.5: Latex report}
This report is very well done. It contains a lot of information. I really liked the analysis of the speed where you showed that NumPy's speed-up factor increased as the grid size increased!

%%%%%%%%%%%%%%%%%%%%%%%%%%%%%%%%%%%%%%%%%%%%%%%%%%%%%%%%%%
\subsection*{Assignment 5.6: More C-interfaces}
Handled further up: \ref{sec:assignment5.2}.

%%%%%%%%%%%%%%%%%%%%%%%%%%%%%%%%%%%%%%%%%%%%%%%%%%%%%%%%%%
\subsection*{Assignment 5.7: Github activity plot}
Not implemented.

\bibliographystyle{plain}
\bibliography{literature}

\end{document}
